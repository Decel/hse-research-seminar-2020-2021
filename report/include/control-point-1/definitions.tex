\section{Основые термины и определения}

\begin{description}
    \item[Метрическое пространство] {\it Метрическим пространством} $(M, \rho)$ называют множество $M$ с функцией $\rho: M \cross M \rightarrow \mathbb{R}_{+}$, называющейся расстояние, для любых $x, y, z \in M$, что:
    \begin{enumerate}
        \item $\rho(x, y) \geq 0$ и $\rho(x, y) = 0$ iff $x = y$
        \item $\rho(x, y) = \rho(y, x)$
        \item $\rho(x, z) \leq \rho(x, y) + \rho(y, z)$
    \end{enumerate}
    \item[Симплициальный комплекс] {\it Симплициальным комплексом} на конечном множестве вершин $M$ называется совокупность $K \subset 2^{M}$ подмножеств множества $M$, удовлетворяющая следующим двум условиям:
    \begin{enumerate}
        \item если $I \in K$ и $J \subset I$, то $J \in K$;
        \item $\varnothing \in K$.
    \end{enumerate}
    \item[Симплекс] {\it Симплексом} называются элементы симплециального комплекса $K$.
    \item[Порядковая переменная] Переменная, которая принимает значения из конечного упорядоченного множества.
    \item[Номинальная переменная] Переменная, которая принимает значения из конечного неупорядоченного множества.
    \item[Кросс-Энтропия] кол-во информации, в среднем, необходимое для идентификации событий из распределения.
    
\end{description}
