\section{Введение} \label{sec:intro}

Современные технологии становятся частью нашей повседневной жизни. 
Каждый из нас в кармане имеет мгновенный доступ к любому человечеством знанию, описанному в интернете; каждый может связаться с другим на каком бы расстоянии мы не были. И без этого уже не представить наш мир. 
Человечеству открываются невероятные горизонты бытия, расширяется кругозор и появляются всё больше возможностей для развития. 
Как личностного развития, так и образовательного.

Конечно, это повлияло на все сферы нашего общества. 
Образование не исключение, здесь мы видим множество изменений. 
Учителя имеют возможность проводить интерактивные уроки с помощью красочных презентаций и образовательных фильмов, дети могут достать любую законную литературу, администрация же подстраивается под новые реалии и организует онлайн классы. 
Почти в каждой школе есть свой Wi-Fi и компьютерный класс. 
Оценить, насколько изменения в образовании положительно влияют на общий процесс образования я попытаюсь с помощью современных технологий, методами машинного обучения и топологического анализа данных, на основе опросов, проведенных социологами в 2020 году. 
Основная задача данных опросов собрать данные об удовлетворенности учителей, школьников и администрации школы нововведениями, связанными с усовершенствованием технологий и мира.

В рамках данного проекта я буду искать интересную информацию про то, какое отражение возымело бурное развитие IT технологий на среднее образование с точки зрения школьников, учителей и сотрудников школьной администрации.
Для этого я изучу основы топологии, которая дает теоретическое обоснование алгоритмам понижения размерности данных UMAP \cite{umap} и Mapper \cite{mapper}, и изображу форму данных, что должно помочь в их кластеризации.
Также, я сравню результаты топологических алгоритмов с алгоритмами классического машинного обучения t-SNE \cite{tsne} и PCA.
В результате надеюсь увидеть некоторые не очевидные зависимости между ответами людей.


\section{Обзор источников}

Нашим основным источником информации касательно топологии является \cite{temp}.
В этой книге рассматриваются темы вроде симплециальных комплексов, сиплексов, нерв-теоремы на уровне, достаточном для понимания работы алгоритмов топологического анализа данных UMAP, Mapper.
Для задачи препроцессинга данных мы ориентировались на устоявшиеся практики современной Data Science, подробно рассмотренные в \cite{feature-eng, feature-eng-sel}.