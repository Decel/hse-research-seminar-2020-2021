\section{Заключение}

В результате проведенных нами вычислительных экспериментов, мы установили, что ответы респондентов не зависят от пола.
Также, мы заметили, что должность респондентов, являющихся сотрудниками школьной администрации, не влияет на их ответы, что является индикатором того, что все сотрудники осведомлены о процессах в своей школе одинаково широко.
Тоже самое можно сказать и про школьников, что, в целом, было нами ожидаемо.

В тоже время, разные алгоритмы кластеризации давали очень разный результат на одном и том же наборе данных.
Алгоритм UMAP каждый раз находил больше кластеров данных, чем, например, PCA, который мог не найти ни одного кластера во входных данных.
Однако, стоит отметить, что ни один из найденных с помощью UMAP кластеров не соответствовал нашим гипотезам, поэтому причина их наличия в данных остается открытым вопросом.