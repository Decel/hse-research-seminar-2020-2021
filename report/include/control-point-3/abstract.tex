\section{Реферат}

Наша работа была мотивированна изучением молодой и активно развивающейся областью топологии -- топологического анализа данных (TDA), возникшая посредством множества работ в вычислительной геометрии и прикладной топологии. 
В основе TDA лежит идея о том, что топология и геометрия обеспечивают крайне эффективный подход к получению надежной качественной и, иногда, количественной информации о структуре данных.
TDA стремится предоставить достаточно обоснованные математические, статистические и алгоритмические методы для вывода, анализа и использования топологических структур, лежащих в основе данных, которые часто представлены в виде точек в метрическом пространстве. 
Целью исследования нашей работы были социальные опросы школьников, руководителей и учителей с помощью которых мы надеялись выявить паттерны посредством использования методов TDA, в частности, UMAP и PCA.
В первом этапе работы, мы занялись очисткой данных от статистических выбросов и изучением методички по топологии, во втором этапе мы сформулировали гипотезы, изучили алгоритмы UMAP и PCA и реализовали их на наших, уже очищенных датасетах, в третьем этапе мы занялись обоснованием корректности нашей работы.
Результаты нашей работы показали, что ответы участников опроса не зависели от пола, класс учеников не влиял на их мнение и распределение ответов сотрудников школьной администрации не зависит от должности.
