\section{Основная часть}

Начальным этапом проекта было ознакомление с предоставленными нам данными, полученных в результате опроса большого количества учеников, учителей и членов школьных администраций, проведенного в 2020 году.
В анкетах содержалось большое количество вопросов касательно общего влияния цифровых устройств и IT инфрастуктуры школы на организацию учебного процесса с точки зрения различных его участников: школьников, преподавателей, директоров.

На этом этапе перед нами стояла задача предварительной обработки полученных от социологов данных, их нормализация и представление в виде облака точек в пространстве достаточно большой размерности.
Из-за своей природы, полученные нами данные являлись весьма не структурированными и нуждались в дополнительной обработке.
Так, например, были отброшены все ответы на вопросы, которые были даны в свободной формы: тематика секции школьника, названия онлайн-сервисов, которыми пользуются ученики.
Среди оставшихся данных были отсеяны очевидные статистические выбросы, такие как противоречивые ответы на вопросы, ответы, в которых почти на все вопросы был дан ответ с одинаковой нумерацией (например, школьник на все вопросы отвечал \enquote{Да} или \enquote{1}).
Над оставшимся массивом данных уже можно было проводить преобразования для отображения ответов каждого респондента в пространство $\{0, 1\}^{m}$.
Для этого мы воспользовались тремя разными алгоритмами: {\bf Ordinal Encoding}, {\bf One-Hot Encoding} и {\bf Dummy Variable Encoding}.

\subsection{Ordinal Encoding}

В случае алгоритма {\bf Ordinal Encoding}, каждой уникальной категории присваивается уникальное целое число.
Например, категории \enquote{Да}, \enquote{Нет} и \enquote{Затрудняюсь ответить} могут быть закодированы через последовательность 1, 2, 3.
Поскольку такое кодирование является весьма естественным и легко обратимым, мы попробовали применить этот алгоритм для кодирования наших данных, однако быстро отказались от этой идеи.
Поскольку такое кодирование не является бинарным, его результаты придется дополнительно нормировать, однако мы не сможем избавиться от относительного порядка на ответах (ответ \enquote{Нет} будет считаться более важным, чем ответ \enquote{Да}).
Таким образом, в текущем виде данный алгоритм нам не подходит.

\subsection{One-Hot Encoding}

Как уже было сказано, нам важно сохранить отсутствие относительного порядка между ответами респондентов, чтобы не вводить будущую модель в заблуждение.
В таких случаях, обычно, применяют алгоритм {\bf One-Hot Encoding}, который преобразует упорядоченные данные в неупорядоченные посредством удаления каждой целочисленной категории и присваивания ей некоторого бинарного значения за каждое уникальную целочисленную категорию этой переменной \cite{feature-eng}.
То есть, полученные значения категорий 1, 2, 3 из результата работы предыдущего алгоритма, раскроются в следующую матрицу:
\[
    A = \begin{pmatrix}
        1 & 0 & 0 \\
        0 & 1 & 0 \\
        0 & 0 & 1
    \end{pmatrix}.
\]
Такой алгоритм кодирования данных прекрасно подходит для наших нужд.

\subsection{Dummy Variable Encoding}

Можно заметить, что в примере из предыдущего абзаца мы храним лишнюю информацию.
В самом деле, нам совершенно не нужен третий столбец матрицы $A$, поскольку из матрицы
\[
    A^{\prime} = \begin{pmatrix}
        1 & 0 \\
        0 & 1 \\
        0 & 0
    \end{pmatrix}
\]
однозначно восстанавливается принадлежность объекта к какой-то категории.
То есть, если человек не ответил \enquote{Да} и не ответил \enquote{Нет}, то мы сразу же делаем вывод, что он затрудняется ответить.
Помимо этого, в некоторых случаях, кодирование через второй алгоритм может сделать матрицу вырожденной \cite{feature-eng-sel}, что плохо сказывается на эффективности и точности некоторых алгоритмов классического машинного обучения (таких как линейная регрессия).
В итоге, для наших наборов данных мы применили именно этот алгоритм с целью приведения ответов респондентов к булевым $m$-мерным векторам для последующего применения алгоритмов топологического анализа данных.