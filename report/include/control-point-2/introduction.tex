\section{Введение} \label{sec:intro}

В первом этапе нашей работы мы изучили методы {\bf Ordinal Encoding}, {\bf One-Hot Encoding} и {\bf Dummy Variable Encoding} и воспользовались ими на наших данных, после очистки от неоднозначных статистических выбросов (на подобии тематики секции, названий онлайн-сервисов, которые использовали ученики.). Так же мы убрали из данных противоречивые ответы на вопросы, вопросы, ответы, в которых почти на все вопросы был дан ответ с одинаковой нумерацией (например, школьник на все вопросы отвечал \enquote{Да} или \enquote{1}). 
Во втором этапе нашей работы мы занялись изучением алгоритмов {\bf UMAP} и {\bf PCA}, и воспользовались ими на наших данных.Для достижения этой цели, нам нужно было написать код, который берет обработанный массив данных, представленный матрицей $A \in \mathbb{R}^{m \times n}$ со стандартизированными столбцами (из каждого вычли его среднее, и поделили на дисперсию).
Чтобы упростить читаемость отчета и работу с исходными данными, мы закодировали все \enquote{содержательные} вопросы --- такие вопросы, по которым очень сложно делать хорошие предсказания (например, ввиду их большого количества), или те вопросы, по которым мы не будем делать предсказания и красить точки в пространстве.
Также мы выделили несколько гипотез относительно данных, которые хотели проверить:
\begin{enumerate}
    \item Распределение ответов на опрос не зависит от гендера участников.
    \item Распределение ответов учеников на опрос не зависит от класса, в котором ученик обучается.
    \item Распределение ответов членов школьной администрации зависит от должности.
\end{enumerate}
В третьем этапе нашей работы мы занялись изучением вопроса предсказания школы человека в зависимости от его ответов, однако не получили результатов достойных внимания, но мы так же занялись формальным обоснованием нашей методологии и перечитали достаточно большой объем литературы для достижения этой цели.
\end{document}
