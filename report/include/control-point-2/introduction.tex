\section{Введение} \label{sec:intro}

Следующим этапом моей работы над проектом были непосредственные манипуляции с данными с целью выявить в них какие-то закономерности.
Для этого, пока что, были рассмотрены алгоритмы UMAP \cite{umap} и PCA \cite{pca}.
Для достижения этой цели, нужно было написать код, который берет обработанный массив данных, представленный матрицей $A \in \mathbb{R}^{m \times n}$ со стандартизированными столбцами (из каждого вычли его среднее, и поделили на дисперсию).
Чтобы упростить читаемость отчета и работу с исходными данными, я закодировал все \enquote{содержательные} вопросы --- такие вопросы, по которым очень сложно делать хорошие предсказания (например, ввиду их большого количества), или те вопросы, по которым мы не будем делать предсказания и красить точки в пространстве.
Также я выделил несколько гипотез относительно данных, которые хотел проверить:
\begin{enumerate}
    \item Распределение ответов на опрос не зависит от гендера участников.
    \item Распределение ответов учеников на опрос не зависит от класса, в котором ученик обучается.
    \item Распределение ответов членов школьной администрации зависит от должности.
\end{enumerate}