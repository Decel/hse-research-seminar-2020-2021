\section{Основые термины и определения}

\begin{description}
    \item[Симплициальный комплекс] {\it Симплициальным комплексом} на конечном множестве вершин $M$ называется совокупность $K \subset 2^{M}$ подмножеств множества $M$, удовлетворяющая следующим двум условиям:
    \begin{enumerate}
        \item если $I \in K$ и $J \subset I$, то $J \in K$;
        \item $\varnothing \in K$.
    \end{enumerate}
    \item[Симплекс] {\it Симплексом} называются элементы симплециального комплекса $K$.
    \item[Порядковая переменная] Переменная, которая принимает значения из конечного упорядоченного множества.
    \item[Номинальная переменная] Переменная, которая принимает значения из конечного неупорядоченного множества.
\end{description}